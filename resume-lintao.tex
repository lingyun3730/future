% !TEX TS-program = xelatex
% !TEX encoding = UTF-8 Unicode
% !Mode:: "TeX:UTF-8"
\documentclass{resume}
\usepackage{zh_CN-Adobefonts_external} % Simplified Chinese Support using external fonts (./fonts/zh_CN-Adobe/)
\usepackage{linespacing_fix} % disable extra space before next section
\usepackage{cite}
\usepackage[colorlinks,linkcolor=blue]{hyperref}

\begin{document}
\pagenumbering{gobble} % suppress displaying page number

\name{Lintao Dang}
\centerline{Software Engineer}
\vspace{2ex}
\contactInfo{lingyun3730@163.com}{(+86) 15316768729}{Shanghai}

\section{\faCogs\ Profile}
\begin{itemize}[parsep=0.5ex]
  \item \textbf{Skilled in} : Data Structure, Algorithms, Java, Spring-boot, Multi-threading, SQL
  \item \textbf{Familiar with} : Raft Consensus, Hadoop, Spark, etcd, Kafka, Kubenetes
\end{itemize}

\section{\faUsers\ Work Experience}
\datedsubsection{\textbf{eBay}}{Apr. 2019 -- Present}
Software Engineer - Payments
\begin{itemize}
  \item Financial Report Subsystem
  \begin{itemize}
    \item[。] Designed and implemented a platform within Financial Accounting System for querying, analyzing, and generating real-time/batch financial reports.
    \item[。] Utilized a Fail-Over multithreading framework to accelerate the report generation process.
  \end{itemize}

  \item ETL Processing into Hadoop
  \begin{itemize}
    \item[。] Performed ETL processes on payment accounting data, extracting, transforming, and loading it.
    \item[。] Using Kafka as a message middleware, persisted data into Hadoop HDFS in parquet format.
    \item[。] Created HIVE table schemas, implemented partitioning, and facilitated HIVE/Spark SQL queries.
    \item[。] Established batch jobs for periodic merging of Parquet data files and reconciliation.
  \end{itemize}

  \item Big Data Solution for Financial Report Generation
  \begin{itemize}
    \item[。] Leveraging HIVE table as a data source, I employed Scala-based Spark programming to generate aging reports for payment accounts.
    \item[。] Significantly enhanced the efficiency of report generationm by performance tuning.
  \end{itemize}

  \item Fund Management Platform
  \begin{itemize}
    \item[。] Contributed to a rule-based fund allocation platform, ensuring atomicity, real-time processing, and idempotence.
  \end{itemize}

  \item FX Platform Setup
  \begin{itemize}
    \item[。] Played a key role in setting up the FX Center, integrating exchange rates from various vendors.
    \item[。] Established the FX Core platform for financial-related FX calculations.
  \end{itemize}

\item Other Contributions
  \begin{itemize}
    \item[。] Collaborated with finance/accounting colleagues to develop the Easy Finance Tool, facilitating convenient and accurate accounting proposals.
    \item[。] Participated in the eBay Financial Accounting System migration, handling balance reconciliation during system transitions.
    \item[。] Project driving, cross-domain communication, E2E test automation, etc.
  \end{itemize}
\end{itemize}

\datedsubsection{\textbf{Morgan Stanlay}}{Jul. 2018 -- Sep. 2018}
Software Engineer - Summer Intern

\section{\faGraduationCap\ Education}
\datedsubsection{\textbf{Shanghai Jiaotong University}}{Sep. 2016 -- Mar. 2019}
\textit{Electronic and Communication Engineering}, Master Degree
\\ \textbf{GPA 3.68/4.0}, awarded the National Postgraduate Scholarship in 2018, secured the second prize in the 2016 National Postgraduate Mathematical Contest in Modeling.
\datedsubsection{\textbf{Tongji University}}{Sep. 2012 -- Jul. 2016}
\textit{Electronic and Communication Engineering}, Bachelor Degree
\\ \textbf{GPA 4.8/5.0}, Received the title of Outstanding Graduate of Shanghai in 2016

\name{党林涛}

% {E-mail}{mobilephone}{homepage}
% be careful of _ in emaill address
\contactInfo{(+86) 153-1676-8729}{lingyun3730@163.com}{软件工程师}
% {E-mail}{mobilephone}
% keep the last empty braces!
%\contactInfo{xxx@yuanbin.me}{(+86) 131-221-87xxx}{}

% \section{\faCogs\ IT 技能}
\section{专业技能}
% increase linespacing [parsep=0.5ex]
\begin{itemize}[parsep=0.2ex]
  \item \textbf{支付}: 熟悉支付记账系统架构,以及相关业务包括支付记账,对账,报表生成,以及换汇FX等。
  \item \textbf{数据结构与算法}: 熟练掌握基础数据结构与算法。
  \item \textbf{编程语言}: 掌握Java基础, 微服务框架 Springboot, Java多线程技术等。
  \item \textbf{分布式技术}: 熟悉Raft分布式系统共识协议,日常使用etcd作为分布式系统强一致Key-Value服务注册工具。
  \item \textbf{大数据}: 熟悉Spark编程模型,善于Spark Job性能调优。
  \item \textbf{消息中间件}: 熟悉消息中间件Kafka基本架构和使用。
  \item \textbf{数据库}: 掌握Oracle数据库查询优化,并使用缓存数据库Redis。
%\item \textbf{其他}: 熟悉Kubenetes 基本原理,搭建与使用。
\end{itemize}
% \end{itemize}

\section{工作经历}
\datedsubsection{\textbf{亿贝软件工程(上海)有限公司 | eBay}, Software Engineer - Payments}{2019.04 - 至今}
\begin{itemize}
  \item \textbf{FAS支付记账系统财务报表子系统设计与实现}
    \begin{itemize}
      \item[。] 财务报表子系统是从FAS支付记账系统中查询,分析,聚合,处理用户交易记录信息,并从不同维度实时或者批量生成财务团队需要的报表的平台系统。
      \item[。] 根据财务团队的业务需求,在学习并理解支付记账系统复杂业务逻辑的基础上进行后台开发,后台部分采用支持Fail-Over的Java多线程框架,搭建etcd作为服务注册工具,充分利用集群资源,保证报表生成的效率和准确度。
      \item[。] 负责后台数据与前端对接的Web Service开发,采用基于OIDC的Sail Point身份管理使系统支持基于权限控制的单点登录SSO。
    \end{itemize}
  \item \textbf{支付记账数据 ETL 持久化到Hadoop}
    \begin{itemize}
      \item[。] 对线上数据做抽取,转化和加载,通过Kafka消息中间件,最后将数据以parquet格式持久化到Hadoop HDFS,创建HIVE table schema并做Partition,支持HIVE SQL数据查询。
      \item[。] 创建 batch job 定期对Parquet数据文件合并,并做数据recon。
    \end{itemize}
  \item \textbf{财务报表生成的大数据解决方案}
    \begin{itemize}
      \item[。] 基于HIVE table的数据源,采用基于Scala的Spark 编程来生成支付账户的账龄报告,性能调优显著提升了报告的生成时效。
    \end{itemize}
  \item \textbf{支付系统资金管理平台}
    \begin{itemize}
      \item[。] 根据不同业务需求,支持rule based资金调拨,保证资金调拨的原子性,实时性和幂等性。
    \end{itemize}
  \item \textbf{支付换汇平台搭建}
    \begin{itemize}
      \item[。] 参与搭建FX Center,集成不同Vendor提供的汇率,为公司所有FX换汇业务提供相应的汇率。
      \item[。] 负责FX Core换汇平台搭建,集成公司所有换汇业务的财务相关FX 计算。
    \end{itemize}
  \item \textbf{其他}
    \begin{itemize}
      \item[。] 领导多个内部项目的开发,包括需求讨论,架构设计,开发,自测,Code Review,E2E,并且快速响应功能上线后出现的故障并及时修复,撰写或完善开发文档等。
      \item[。] 与Finance/Accounting同事沟通日常需求,了解他们的痛点后开发了Easy Finance Tool帮助他们更方便准确地在平台上Propose记账需求,支持Approval功能并满足财务Compliance要求。
      \item[。] 2019年8月-2020年3月参与eBay FAS 支付记账系统迁移工作,开发程序对部分账户的余额从老系统到新系统迁移时的对账。
    \end{itemize}
\end{itemize}

\datedsubsection{\textbf{摩根士丹利管理服务(上海)有限公司}, Software Engineer - Summer Intern}{2018.07 - 2018-09}

% \section{\faGraduationCap\ 教育背景}
\section{教育背景}
\datedsubsection{\textbf{上海交通大学},电子与通信工程,\textit{硕士}}{2016.09 - 2019.03}
\ \textbf{GPA: 3.68/4.0}, 获得2018年研究生国家奖学金,2016年全国研究生数学建模竞赛二等奖。
\datedsubsection{\textbf{同济大学},通信工程,\textit{本科}}{2012.09 - 2016.07}
\ \textbf{GPA: 4.8/5.0}, 获得2013,2014,2015年校级学习奖学金,2016年上海市优秀毕业生。

\end{document}
